% Paper template for TAR 2022
% (C) 2014 Jan Šnajder, Goran Glavaš, Domagoj Alagić, Mladen Karan
% TakeLab, FER

\documentclass[10pt, a4paper, croatian]{article}

\usepackage{tar2023}

\usepackage[provide=*]{babel}

\usepackage[utf8]{inputenc}
\usepackage[pdftex]{graphicx}
\usepackage{booktabs}
\usepackage{amsmath}
\usepackage{amssymb}

\title{Nadopunjavanje slike korištenjem difuzijskih modela}

\name{Martin Bakač, Mislav Đomlija, Ivan Kapusta, Maksim Kos, Antonio Lukić, Jerko Šegvić}

\address{
University of Zagreb, Faculty of Electrical Engineering and Computing\\
Unska 3, 10000 Zagreb, Croatia\\ 
}
          
         
\abstract{
Cilj nadopunjavanja slike je nedostajuće piksele slike zamijeniti čim uvjerljivijim sadržajem. U projektu smo opisali i implementirali model za nadopunjavanje slika temeljen na difuziji. Model je treniran na skupu podataka MNIST i ostvarili smo \textbf{TODO: ovakve i onakve rezultate}.
}

\begin{document}

\maketitleabstract

\section{Nadopunjavanje slike}
Nadopunjavnje slike (engl. \emph{image inpainting}) je postupak u kojem nadopunjavamo segmente slike koji nedostaju. Nadopuna mora biti 
semantički smislena i u skladnom odnosu s poznatim dijelom slike. Zbog ovih čvrstih zahtjeva, modeli za nadopunjavanje slike moraju imati 
snažne generativne mogućnosti. Generativni difuzijski modeli, uz dobre generativne sposobnosti, omogućuju i manipulaciju latentnog prostora 
za ugradnju semantičke informacije. Popularni modeli za ovaj zadatak su još i generativne suparničke mreže 
(engl. \emph{generative adverserial networks}) i autoregresivni modeli. Problem takvih arhitketura je da moraju također naučiti i distribuciju
dijelova slike koji nedostaju. U praksi takvi modeli imaju problema sa neobičnim prazninama u slici ili sa prazninama s kojima se nisu
susreli prilikom treniranja


\section{Difuzijski modeli}
Difuzijski modeli su su klasa generativnih modela s latentnim varijablama. Ideja difuzijskih modela je simulirati proces reverzne difuzije 
i naučiti model da iz poznate latentne distribucije konstruira elemente iz distribucije podataka. Svaki difuzijski model se sastoji od tri
osnovna elementa: unaprijedni proces, reverzni proces i strategija uzrokovanja
\section{RePaint model}

\section{Rezultati}

\section{Zaključak}

\textbf{TODO: ovo bude GPT na kraju napisal}

%\section*{Acknowledgements}

\bibliographystyle{tar2023}
\bibliography{tar2023} 

\end{document}

